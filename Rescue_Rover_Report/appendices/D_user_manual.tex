% appendices/D_user_manual.tex - Appendix D: User Manual
% =======================================================

\chapter{User Manual}
\label{app:user-manual}

This appendix provides practical guidance for operating the Rescue Rover system, from initial setup through routine maintenance.

% --------------------------------------------------------
\section{Quick Start}
\label{sec:quick-start}

\textbf{Step 1: Power the Rover}
\begin{itemize}
    \item Connect a fully charged LiPo battery (11.1V 3S) to the L298N power terminals.
    \item Verify the power LED on L298N illuminates.
    \item The ESP32-S3 onboard LED should turn on within 2 seconds.
\end{itemize}

\textbf{Step 2: Verify Hardware}
\begin{itemize}
    \item Confirm the camera ribbon cable is securely seated in the FPC connector.
    \item Place the rover on a flat surface with unobstructed wheels.
    \item Ensure the ultrasonic sensor at the front is clean and unblocked.
\end{itemize}

\textbf{Step 3: Start the Host Application}

\vspace{0.5em}
\begin{lstlisting}[language=bash]
cd RoverInterface
python -m venv .venv
source .venv/bin/activate   # Windows: .venv\Scripts\activate
pip install -r requirements.txt
python app.py
\end{lstlisting}

\textbf{Step 4: Connect to the Dashboard}
\begin{itemize}
    \item Open a browser to \url{http://localhost:8080}
    \item Wait for the video feed to appear (may take 5--10 seconds on first connection).
    \item Verify the ``Connection'' indicator shows green.
\end{itemize}

% --------------------------------------------------------
\section{Dashboard Interface}
\label{sec:dashboard-interface}

The operator dashboard is divided into four main areas:

\begin{description}[style=nextline]
    \item[Video Panel (Left)]
    Live MJPEG stream from the rover camera. Displays frame rate overlay in the top corner. Click to toggle full-screen mode.
    
    \item[Telemetry Panel (Top Right)]
    Real-time status indicators:
    \begin{itemize}
        \item \textbf{Battery}: Green ($>$11V), Yellow (10--11V), Red ($<$10V, land immediately)
        \item \textbf{Distance}: Ultrasonic reading in centimeters
        \item \textbf{Cloud Status}: Green (connected), Yellow (lagging), Red (offline)
    \end{itemize}
    
    \item[Mission Log (Center Right)]
    Scrollable event log showing AI decisions, hazard alerts, and system messages.
    
    \item[Control Panel (Bottom Right)]
    Manual control buttons and evidence capture trigger.
\end{description}

% --------------------------------------------------------
\section{Firmware Upload}
\label{sec:firmware-upload}

\textbf{Requirements}
\begin{itemize}
    \item Arduino IDE 2.0+ (or PlatformIO)
    \item ESP32 Board Support Package installed
    \item USB-C data cable (not charge-only)
\end{itemize}

\textbf{Arduino IDE Configuration}

Open \texttt{Tools} menu and set:
\begin{itemize}
    \item Board: \texttt{ESP32S3 Dev Module}
    \item PSRAM: \texttt{OPI PSRAM}
    \item Flash Mode: \texttt{QIO 80MHz}
    \item Partition Scheme: \texttt{Huge APP (3MB No OTA/1MB SPIFFS)}
    \item Upload Speed: \texttt{921600}
\end{itemize}

\textbf{Upload Procedure}
\begin{enumerate}
    \item Connect the ESP32-S3 to your computer via USB-C.
    \item Select the correct COM port (look for ``CP2102'' or ``CH340'' in device name).
    \item Click the Upload button.
    \item If upload fails, hold the BOOT button on the board, click Upload again, then release BOOT when progress begins.
\end{enumerate}

% --------------------------------------------------------
\section{Troubleshooting}
\label{sec:troubleshooting}

\begin{longtable}{p{4cm}p{10cm}}
    \toprule
    \textbf{Symptom} & \textbf{Solution} \\
    \midrule
    \endfirsthead
    \toprule
    \textbf{Symptom} & \textbf{Solution} \\
    \midrule
    \endhead
    
    No video feed & 
    Check camera ribbon cable is fully inserted. Verify PSRAM is enabled in Arduino IDE. Try reducing resolution to QQVGA in \texttt{CameraModule.cpp}. \\
    
    Motors not responding & 
    Check L298N power connections. Verify enable jumpers (ENA/ENB) are installed. Test motors directly with 5V to confirm they work. \\
    
    WiFi not connecting & 
    Verify SSID and password in \texttt{RescueRobot.ino}. Confirm router is 2.4GHz (ESP32 does not support 5GHz). \\
    
    Dashboard shows ``Disconnected'' & 
    Verify rover and host are on the same network. Check that Gateway device is plugged in and receiving packets. \\
    
    Camera initialization fails & 
    Reduce XCLK frequency to 8MHz in camera config. Verify all camera pins are correctly connected. Check PSRAM is functional. \\
    
    Intermittent control & 
    Move closer to router. Reduce WiFi interference (microwaves, other 2.4GHz devices). Verify ESP-NOW peer MAC address matches. \\
    
    \bottomrule
\end{longtable}

\textbf{LED Status Indicators}

\begin{table}[h!]
    \centering
    \begin{tabular}{lll}
        \toprule
        \textbf{LED} & \textbf{Pattern} & \textbf{Meaning} \\
        \midrule
        L298N Power LED    & Solid      & Power connected \\
        ESP32-S3 LED       & Solid      & Booting or idle \\
        ESP32-S3 LED       & Fast blink & Streaming video \\
        ESP32-S3 LED       & Slow blink & Connecting to WiFi \\
        ESP32-S3 LED       & Off        & No power or boot failure \\
        \bottomrule
    \end{tabular}
\end{table}

% --------------------------------------------------------
\section{Safety Guidelines}
\label{sec:safety-guidelines}

\textbf{Battery Safety}
\begin{itemize}
    \item Never leave a charging LiPo battery unattended.
    \item Do not puncture, crush, short-circuit, or expose to extreme heat.
    \item Dispose of swollen or damaged batteries at proper recycling facilities.
    \item For long-term storage, charge to 3.8V per cell (approximately 40--60\% capacity).
\end{itemize}

\textbf{Operational Safety}
\begin{itemize}
    \item Keep fingers and loose clothing away from moving wheels.
    \item Do not operate on stairs, ledges, or unstable surfaces.
    \item Avoid extended full-power operation to prevent motor overheating.
    \item Always test in a controlled environment before field deployment.
\end{itemize}

\textbf{Electrical Safety}
\begin{itemize}
    \item Never modify wiring while the system is powered.
    \item Protect electronics from water, dust, and static discharge.
    \item Ensure all wire connections are insulated with heat shrink or electrical tape.
\end{itemize}

% --------------------------------------------------------
\section{Maintenance Schedule}
\label{sec:maintenance}

\textbf{After Each Mission}
\begin{itemize}
    \item Download mission logs from the dashboard.
    \item Charge battery to storage voltage (3.8V/cell) if not operating within 24 hours.
    \item Inspect chassis for physical damage.
\end{itemize}

\textbf{Weekly}
\begin{itemize}
    \item Clean camera lens with a microfiber cloth.
    \item Check wheel attachment screws for tightness.
    \item Inspect all wiring for loose connections or fraying.
\end{itemize}

\textbf{Monthly}
\begin{itemize}
    \item Clean chassis internals with compressed air.
    \item Test battery capacity with a LiPo checker.
    \item Check for firmware updates in the project repository.
\end{itemize}
