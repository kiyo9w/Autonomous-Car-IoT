% chapters/02_system_architecture.tex - Chapter 2: System Architecture
% =====================================================================

\chapter{System Architecture}
\label{chap:system-architecture}

This chapter presents the overall system architecture of the Rescue Rover. The design follows a distributed intelligence model spread across three physical tiers and one cloud tier \cite{ermacora2013cloudemergency}. This hybrid approach allows high performance AI inference without compromising local real time control \cite{dong2025edgerobotics}.

% --------------------------------------------------------
\section{Architectural Overview}
\label{sec:arch-overview}

The system consists of four distinct processing nodes connected through a hierarchy of communication channels.

\begin{enumerate}
    \item \textbf{The Rover (Edge Tier 1)}: ESP32-S3 microcontroller \cite{espressif2023esp32s3}. Handles motor control, sensor readings, and video capture. It has no AI capability but provides 10ms-response safety reflexes.
    \item \textbf{The Gateway (Edge Tier 2)}: ESP32 bridge. Translates between the Rover's wireless ESP NOW protocol and the Host's USB serial connection.
    \item \textbf{The Host Computer (Edge Tier 3)}: MacBook Pro. Runs the "Tactical" AI layer (YOLOv8) for immediate object detection and hosts the operator dashboard. It serves as the local command center.
    \item \textbf{The Cloud Server (Cloud Tier)}: Google Colab via Google Cloud Platform. Runs the "Strategic" AI layer (Qwen2.5-VL-7B) \cite{bai2025qwen2vl} on NVIDIA H100 GPUs. It handles complex scene reasoning that is too heavy for the local host.
\end{enumerate}

% FIGURE: System Overview
\begin{figure}[h!]
    \centering
    \includegraphics[width=\textwidth]{figures/software/system_overview_hybrid.png}
    \caption{High level system architecture showing the four processing nodes and their data links.}
    \label{fig:system-overview}
\end{figure}

\begin{table}[h!]
    \centering
    \caption{Device roles and capabilities}
    \label{tab:device-roles}
    \begin{tabular}{llp{6cm}}
        \toprule
        \textbf{Node} & \textbf{Hardware} & \textbf{Responsibilities} \\
        \midrule
        Rover   & ESP32-S3 + OV2640   & Video capture, motor actuation, reflex safety (sonar) \\
        Gateway & ESP32 WROOM         & Protocol translation (ESP NOW $\leftrightarrow$ Serial) \\
        Host    & Apple Silicon Mac   & Dashboard UI, Tactical AI (YOLO), Command Arbitration \\
        Cloud   & NVIDIA H100 GPU     & Strategic AI (Vision Language Model) \\
        \bottomrule
    \end{tabular}
\end{table}

% --------------------------------------------------------
\section{Hybrid Intelligence Model}
\label{sec:hybrid-intelligence}

The system implements a three layer intelligence architecture that physically separates "reflex" from "reasoning."

% FIGURE: Hybrid Intelligence Pyramid
\begin{figure}[h!]
    \centering
    \includegraphics[width=0.8\textwidth]{figures/software/hybrid_layers_cloud.png}
    \caption{The three layer hybrid intelligence model distributed across Edge and Cloud.}
    \label{fig:hybrid-layers}
\end{figure}

\textbf{Layer 1: Reactive Control (Firmware).} The reactive layer runs on the ESP32-S3 firmware. It handles immediate threat preservation.
\begin{itemize}
    \item \textbf{Input}: Ultrasonic distance sensor.
    \item \textbf{Action}: Hard E STOP if distance $<$ 25cm.
    \item \textbf{Latency}: $<$ 10ms.
    \item \textbf{Reliability}: 100\% (Works even if WiFi/Host fails).
\end{itemize}

\textbf{Layer 2: Tactical Processing (Local Host).} The tactical layer runs on the local Mac using YOLOv8 \cite{jocher2023yolov8} (CoreML). It handles dynamic obstacles.
\begin{itemize}
    \item \textbf{Input}: Video stream (30 FPS).
    \item \textbf{Action}: "Stop for Person", "Avoid Chair".
    \item \textbf{Latency}: $\sim$30ms.
    \item \textbf{Reliability}: High dependency on WiFi video stream.
\end{itemize}

\textbf{Layer 3: Strategic Planning (Cloud H100).} The strategic layer runs on Google Colab using Qwen2.5-VL-7B \cite{huang2023vlmaps}. It handles complex navigation logic.
\begin{itemize}
    \item \textbf{Input}: Single frame snapshot (sampled at 0.5Hz).
    \item \textbf{Action}: "The path is blocked by debris, turn around and try the left door."
    \item \textbf{Latency}: 500-1500ms.
    \item \textbf{Reliability}: Subject to Internet connectivity. Fails gracefully if disconnected.
\end{itemize}

% --------------------------------------------------------
\section{Communication Protocols}
\label{sec:comm-protocols}

The hybrid architecture introduces internet layer communication to the stack.

\begin{table}[h!]
    \centering
    \caption{Communication protocols by link}
    \label{tab:protocol-overview}
    \begin{tabular}{llll}
        \toprule
        \textbf{Link} & \textbf{Protocol} & \textbf{Latency} & \textbf{Data Type} \\
        \midrule
        Rover $\leftrightarrow$ Gateway & ESP NOW \cite{espressif2023espnow,espnow2022comparative} & 2-5 ms & Commands, Telemetry \\
        Rover $\rightarrow$ Host & UDP & 80-120 ms & MJPEG Video Stream \\
        Gateway $\leftrightarrow$ Host & USB Serial & 5 ms & Bridge Data \\
        Host $\leftrightarrow$ Cloud & HTTP (ngrok) & 200-500 ms & JSON Request/Response \\
        \bottomrule
    \end{tabular}
\end{table}

\textbf{Cloud Link (HTTP over ngrok).} The Host communicates with the Cloud Server using standard HTTP POST requests securely tunneled via \texttt{ngrok}. The Host sends a JPEG image and prompts; the Cloud returns a JSON object with navigation instructions. This request response cycle occurs asynchronously to avoid blocking the real time control loop.

% FIGURE: Hybrid Data Flow
\begin{figure}[h!]
    \centering
    \includegraphics[width=\textwidth]{figures/software/hybrid_data_flow.png}
    \caption{Data flow diagram comparing the high frequency local control loop and the low frequency cloud analysis loop.}
    \label{fig:hybrid-flow}
\end{figure}

% --------------------------------------------------------
\section{Failure Modes and Recovery}

The introduction of a Cloud dependency adds a specific failure mode: Internet Disconnection.

\begin{table}[h!]
    \centering
    \caption{Hybrid Failure Matrix}
    \label{tab:hybrid-failure}
    \begin{tabular}{lp{5cm}p{5cm}}
        \toprule
        \textbf{Failure} & \textbf{Impact} & \textbf{Recovery} \\
        \midrule
        Local WiFi Loss & Video/Control lost & Rover stops (Heartbeat failsafe) \\
        Internet Loss & Strategic IQ lost & Fallback to "Tactical Only" mode (YOLO remains active) \\
        Cloud Latency Spike & Old commands & Arbiter ignores stale Cloud commands ($>$ 2s old) \\
        \bottomrule
    \end{tabular}
\end{table}

The design ensures that \textbf{internet loss does not compromise safety}. The Rover can still be controlled manually or stop automatically for obstacles via Layer 1/2, even if the "Brain" (Layer 3) is offline.
